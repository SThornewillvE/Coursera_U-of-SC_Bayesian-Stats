\documentclass[journal, a4paper]{IEEEtran}

% Import packages
\usepackage{graphicx}
\usepackage{url}
\usepackage{amsmath}

% Create New Commands

\begin{document}
	\title{Bayesian Stats Techniques and Models Capstone:\\ Weight Loss Analysis}
	\author{Simon Thornewill von Essen\\ Dates: 07/02/2020 - 10/02/2020
	\thanks{}}
	\markboth{}{}
	\maketitle


\begin{abstract}

Abstract Here.

\end{abstract}

\section{Introduction}
	\IEEEPARstart{I}{ntroduction} here.
	
* Clearly identify your problem and the specific question you wish to answer.

* Justify why the data chosen provide insight to answering your question.

* Describe how the data were collected.

* Describe any challenges relating to data acquisition/preparation (such as missing values, errors, etc.).
	
\subsection{The Data}

\section{Data Exploration}

* Graphically explore the data using plots that could potentially reveal insight relating to your question.

\section{Modeling}

\subsection{Model Postulation}

* Justify why the model you chose is appropriate for the type of data you have.

* Describe how the model is well suited to answer your question.

* Identify how inference for parameters in the model will provide evidence relating to your question.

* Write the full hierarchical specification of the model.

* Justify your choice of prior distributions.

\subsection{Model Fitting}

* Fit the model using JAGS and R.

\subsection{Convergence Diagnostics}

* Assess MCMC convergence. It is not necessary to include trace plots or other diagnostics in the report. Commenting on the results of your diagnostics is sufficient.

* Check that modeling assumptions are met (e.g., residual analyses, predictive performance, etc.).

* Decide if your model is adequate. Postulate and fit at least one alternative model and assess which is best for answering your question. If neither is adequate, report that and move on.

\section{Results}

* Provide relevant posterior summaries.

* Interpret the model results in the context of the problem.

* Use the results to reach a conclusion.

* Acknowledge shortcomings of the model or caveats for your results.

\section{Conclusion}

Conclusion here.

\newpage
\onecolumn

\section{Appendix}

Appendix here.

\end{document}